\documentclass{report}

\usepackage{tikz}
\usepackage{graphicx}
\usepackage{amsmath}
\usepackage{palatino}

\usepackage[style=authoryear]{biblatex}
\addbibresource{references.bib}

\newcommand{\totd}[2]{\frac{D #1}{D #2}}
\newcommand{\pd}[2]{\frac{\partial #1}{\partial #2}}
\newcommand{\bu}{\mathbf{u}}
\newcommand{\ub}{\bar{u}}
\newcommand{\tavg}[1]{\langle #1 \rangle}
\newcommand{\bavg}[1]{\langle \overline{#1} \rangle}
\newcommand{\brunt}{Brunt-V\"ais\"al\"a\ }
%%% Local Variables:
%%% mode: latex
%%% TeX-master: "report"
%%% End:


\title{Moist Convection ML}
\author{Noah D. Brenowitz}
\date{\today}

\begin{document}
\maketitle
\tableofcontents

\chapter{Introduction}

\chapter{Training Data}

\section{System for Atmospheric Modeling}
\label{sec:sam}

The System for Atmospheric Modeling (SAM) uses a different set of prognostic
equations \autocite{Khairoutdinov2003}. These are given by

\begin{subequations}
  \label{eq:sam}
  \begin{align}
    &\totd{\bu}{t} + ( f  + \beta y) \mathbf{k}\times \bu = - \nabla \phi \\
    &\totd{w}{t} = -\nabla \phi + g B \label{eq:w-full}\\
    &\totd{h_L}{t}  = Q_{rad}  - \frac{1}{\rho_0} \pd{}{z} \left( L_c P_r + L_s P_s + L_s P_g \right)\\
    &\totd{q_T}{t} = -  \dot{q}_p^{\text{micro}} \label{eq:sam-qT}\\
    &\totd{q_p}{t} = \frac{1}{\rho_0} \pd{}{z} \left(  P_r +  P_s + P_g \right) + \dot{q}_p^{\text{micro}}\\
    &\nabla \cdot \bu + \frac{1}{\rho_0} \pd{\rho_0 w}{z}  = 0.
  \end{align}
\end{subequations}
I have neglected the sub-grid-scale flux terms in the formulation above since
they are not relevant to the problem considered in this report. The temperature
and humidity variables introduced in the previous section have been replaced
here by
\begin{description}
\item[Liquid/ice water static energy] $h_L = c_p T +gz - L_c (q_c + q_r) - L_s
  (q_i + q_s + q_g)$,
\item[Total nonprecipitating water] $q_T = q_v + q_c + q_i$, and
\item[Total precipitating water] $q_p = q_r + q_s + q_g$.
\end{description}
Note that $h_L$ is only an appropriate variable when hydrostatic balance or
anelastic equations are satisfied. Otherwise, potential temperature should be
used. There are a number of diagnostic equations which can be solved to
determine the temperature, and all the mixing ratios for the different species
of precipitating and non-precipitating water. Perhaps most importantly, the
water vapor is assumed to be at or below an approximate saturation value so that
\[q_v = \min(q_T, q_{sat}(p_0, T)),\]
where
\[ q_{sat} = q_{sw} (1-\omega_i) + \omega_i q_{si}\]
is a convex combination of the saturation mixing ratios of vapor with respect to
ice and liquid water. 

A consequence of this diagnostic relation is that the latent
heating term due to condensation is not explicitly calculated. Instead, SAM
models the microphysical processes which govern the conversion from condensate
($\text{cloud water} + \text{cloud ice}$) to precipitation ($\text{rain} +
\text{snow} + \text{grauppel}$). These processes are given by
\[
  \dot{q}_p^{\text{micro}} = \text{accretion} - \text{evaporation} + \text{autoconversion}.
\]
A nice paper by \textcite{hernandez-duenas_minimal_2013} contains a more thorough
discussion of a similar set of microphysical equations, which can be helpful for
gaining intuition. Additional details about the diagnostic relations linking the
water species and the prognostic variables $q_T$, $h_L$, and $q_p$ are available
in Appendix \ref{sec:diag-q} and in the original SAM paper by \textcite{Khairoutdinov2003}.

This formulation is convenient for numerical modeling purposes, but
moist-convective processes are often understood as a coupling between dry
dynamics and moist variables because the bouyancy is related primarily to
temperature, rather than moist static energy.
For instance, the circulation due to the monsoons and MJO can be understood
roughly as the response to a imposed heating pattern \autocite{Gill1980}, and
the multiscale theories of Biello and Majda rely upon asympotic analysis of the
latent heating.

\section{Near-global Model Configuration}
\label{sec:model-config}

We configure the model in the same way as \textcite{Bretherton2015}. 


\chapter{Methods}

\section{Post-processing}
\label{sec:post-processing}

The main post processing steps are

\begin{enumerate}
\item Coarse-graining, and 
\item computing the apparent heat and moisture source due to convection.
\end{enumerate}



\section{Maximum Covariance Analysis}
\label{sec:mca}


\printbibliography

\end{document}
